\documentclass[]{tufte-book}

% ams
\usepackage{amssymb,amsmath}

\usepackage{ifxetex,ifluatex}
\usepackage{fixltx2e} % provides \textsubscript
\ifnum 0\ifxetex 1\fi\ifluatex 1\fi=0 % if pdftex
  \usepackage[T1]{fontenc}
  \usepackage[utf8]{inputenc}
\else % if luatex or xelatex
  \makeatletter
  \@ifpackageloaded{fontspec}{}{\usepackage{fontspec}}
  \makeatother
  \defaultfontfeatures{Ligatures=TeX,Scale=MatchLowercase}
  \makeatletter
  \@ifpackageloaded{soul}{
     \renewcommand\allcapsspacing[1]{{\addfontfeature{LetterSpace=15}#1}}
     \renewcommand\smallcapsspacing[1]{{\addfontfeature{LetterSpace=10}#1}}
   }{}
  \makeatother

\fi

% graphix
\usepackage{graphicx}
\setkeys{Gin}{width=\linewidth,totalheight=\textheight,keepaspectratio}

% booktabs
\usepackage{booktabs}

% url
\usepackage{url}

% hyperref
\usepackage{hyperref}

% units.
\usepackage{units}


\setcounter{secnumdepth}{2}

% citations
\usepackage{natbib}
\bibliographystyle{apalike}


% pandoc syntax highlighting

% table with pandoc
\usepackage{longtable,booktabs,array}
\usepackage{calc} % for calculating minipage widths
% Correct order of tables after \paragraph or \subparagraph
\usepackage{etoolbox}
\makeatletter
\patchcmd\longtable{\par}{\if@noskipsec\mbox{}\fi\par}{}{}
\makeatother
% Allow footnotes in longtable head/foot
\IfFileExists{footnotehyper.sty}{\usepackage{footnotehyper}}{\usepackage{footnote}}
\makesavenoteenv{longtable}

% multiplecol
\usepackage{multicol}

% strikeout
\usepackage[normalem]{ulem}

% morefloats
\usepackage{morefloats}


% tightlist macro required by pandoc >= 1.14
\providecommand{\tightlist}{%
  \setlength{\itemsep}{0pt}\setlength{\parskip}{0pt}}

% title / author / date
\title{Exploratieve en Descriptieve Data Analyse}
\author{Gert Janssenswillen, Benoît Depaire}
\date{}

% \usepackage{booktabs}
% \usepackage{tabu}
\titleclass{\subsubsection}{straight}
\titleformat{\subsubsection}%
  [hang]% shape
  {\normalfont\large\itshape}% format applied to label+text
  {\thesubsubsection}% label
  {1em}% horizontal separation between label and title body
  {}% before the title body
  []% after the title body
\hypersetup{colorlinks=true}
\usepackage{booktabs}
\usepackage{longtable}
\usepackage{array}
\usepackage{multirow}
\usepackage{wrapfig}
\usepackage{float}
\usepackage{colortbl}
\usepackage{pdflscape}
\usepackage{tabu}
\usepackage{threeparttable}
\usepackage{threeparttablex}
\usepackage[normalem]{ulem}
\usepackage{makecell}
\usepackage{xcolor}

\begin{document}

\maketitle



{
\setcounter{tocdepth}{1}
\tableofcontents
}

\hypertarget{voorwoord}{%
\chapter*{Voorwoord}\label{voorwoord}}
\addcontentsline{toc}{chapter}{Voorwoord}

Dit boek bevat de lecture notes en tutorials voor het opleidingsonderdeel ``Exploratieve en Descriptieve Data Analyse'' (1ste Ba Handelsingenieur/Handelsingenieur in de Beleidsinformatica) aan de Universiteit Hasselt. De lectures notes dienen ter begeleiding van de hoorcolleges, terwijl de tutorials telkens een vervolg zijn hierop ter voorbereiding van de werkzittingen.

\hypertarget{hoe-de-lecture-notes-te-gebruiken}{%
\section*{Hoe de lecture notes te gebruiken}\label{hoe-de-lecture-notes-te-gebruiken}}
\addcontentsline{toc}{section}{Hoe de lecture notes te gebruiken}

Het idee van de lecture notes is om een begeleidende tekst aan te reiken ter ondersteuning van de slide-decks die gebruikt worden tijdens de hoorcolleges. Deze tekst is ``bullet-point''-gewijs opgebouwd en helpt het verhaal dat tijdens het hoorcollege wordt verteld terug op te roepen. Daarnaast zal er per hoofdstuk ook een \emph{referentielijst} aangereikt worden met werken die de diverse topics in detail uitleggen.

\begin{itemize}
\tightlist
\item
  Neem de lecture notes mee naar het hoorcollege (digitaal of geprint), en gebruik deze om belangrijke aspecten tidjens het hoorcollege te markeren en korte nota's toe te voegen. Ga zeker niet de volledige uitleg van het hoorcollege noteren. Dit is vaak niet mogelijk en indien je er toch in slaagt zal je tijdens het hoorcollege niet in staat zijn geweest om een eerste keer te reflecteren over de leerstof.
\item
  Bestudeer na de les de lecture notes samen met de notities. Controleer of je alles begrijpt en waar nodig noteer je aanvullingen. Probeer een overzicht te verkrijgen van de diverse concepten die je tijdens het hoorcollege bestudeerd hebt en tracht na te gaan hoe je deze inzichten kunt gebruiken voor exploratieve en descriptieve data analyse.
\item
  (optioneel) Lees de bronnen in de referentielijst. Indien er elementen niet duidelijk zijn in je eigen notities of de lecture notes, dan ga je best gericht op zoek naar de antwoorden op je vragen in de referentiewerken.
\end{itemize}

\hypertarget{hoe-de-tutorials-te-gebruiken}{%
\section*{Hoe de tutorials te gebruiken}\label{hoe-de-tutorials-te-gebruiken}}
\addcontentsline{toc}{section}{Hoe de tutorials te gebruiken}

De tutorials zijn een logisch gevolg op de leerstof in het hoorcollege en bereiden je voor op de oefening in de werkzittingen. In de tutorials worden de concepten uit het hoorcollege geïllustreerd in R code. Het is niet enkel de bedoeling de tutorials te lezen, maar ook zelf de voorbeelden uit te proberen in Rstudio. Je vindt de nodige datasets hiervoor telkens terug op Blackboard. Zonder de tutorials grondig te bekijken heeft het geen zin om naar de werkzittingen te komen.

\hypertarget{over-de-auteurs}{%
\section*{Over de auteurs}\label{over-de-auteurs}}
\addcontentsline{toc}{section}{Over de auteurs}

dr. Gert Janssenswillen is academisch medewerker aan de faculteit Bedrijfseconomische Wetenschappen (BINF Business Informatics) van de Universiteit Hasselt. Na het verwerven van zijn diploma Handelsingenieur in de Beleidsinformatica in 2014, behaalde hij in 2019 een PhD in de Bedrijfseconomie aan de Universiteit Hasselt. Tijdens zijn doctoraat ontwikkelde hij de open-source R packages-suite bupaR, welke wereldwijd gebruikt wordt door bedrijven en organisaties voor de analyse van bedrijfsprocessen. Hij spreekt regelmatig op business process management - conferenties, zoals BPM, ICPM, en SIMPDA, alsook R conferenties, zoals useR. Sinds 2019 is hij lid van het organisatie comité van de Europese R User Meeting (eRum).

Prof.~dr. Benoît Depaire is hoofddocent Beleidsinformatica aan de Universiteit Hasselt en lid van de onderzoeksgroep Beleidsinformatica. Zijn onderzoeksinteresse situeert zich rond de topics data mining, data-gedreven procesanalyse en statistiek met een focus op de extractie van bedrijfskundige inzichten uit data. Als voorzitter van het onderwijsmanagementteam voor de opleiding Beleidsinformatica, alsook op basis van zijn jarenlange ervaring als docent, heeft hij een onderwijsexpertise uitgebouwd rond diverse topics zoals projectmanagement, business process management, data analyse en de rol van IT in de moderne bedrijfswereld. Daarnaast houdt hij zich ook bezig met dienstverlening naar de bedrijfswereld toe door middel van gastlezingen, adviesverstrekking en toegepaste onderzoeksprojecten.

\hypertarget{disclaimer}{%
\section*{Disclaimer}\label{disclaimer}}
\addcontentsline{toc}{section}{Disclaimer}

Niets uit deze uitgave mag worden verveelvoudigd, opgeslagen in een geautomatiseerd gegevensbestand en/of openbaar gemaakt in enige vorm of op enige wijze, hetzij elektronisch, mechanisch, door fotokopieën, opnamen of op enige andere manier zonder voorafgaande schriftelijke toestemming van de uitgever.

\hypertarget{lecture-notes-introductiecollege}{%
\chapter{{[}Lecture notes{]} Introductiecollege}\label{lecture-notes-introductiecollege}}

\hypertarget{data-science}{%
\section{Data science?}\label{data-science}}

\hypertarget{verschillende-soorten-van-data-analyse}{%
\subsection{Verschillende soorten van data analyse}\label{verschillende-soorten-van-data-analyse}}

\begin{itemize}
\tightlist
\item
  Er zijn verschillende manieren om data analyse taken te classificeren.
\item
  De classificatie die we hier hanteren is gebaseerd op het doel van de data analyse.
\end{itemize}

\hypertarget{descriptieve-data-analyse}{%
\subsubsection*{Descriptieve data analyse}\label{descriptieve-data-analyse}}
\addcontentsline{toc}{subsubsection}{Descriptieve data analyse}

\begin{itemize}
\tightlist
\item
  Deze analyse focust zich op het beschrijven van de data.
\item
  Deze analyse gaat over het samenvatten van de grote hoeveelheid data in enkele statistische cijfers en grafieken.
\item
  Deze analyse wordt gebruikt als je een grote hoeveelheid data krijgt en je snel inzicht wilt krijgen in de data.
\item
  Voorbeelden:

  \begin{itemize}
  \tightlist
  \item
    Je hebt een dataset met alle studieresultaten van de studenten van 1ste bachelor HI/BI en je wilt weten wat de gemiddelde score is per vak.
  \item
    Je hebt de verkoopscijfers van het afgelopen jaar en je wil weten welke drie producten het beste verkochten (zowel in aantal als in omzet).
  \end{itemize}
\item
  Descriptieve data analyse zegt alleen iets over de realiteit die door de data is beschreven. Je kan \textbf{geen} conclusies trekken die verder reiken dan de geobserveerde data.
\item
  Je kan een descriptieve data analyse vergelijken met het werk van een detective die als taak heeft een beschrijving te maken van de misdaadscene.
\end{itemize}

\hypertarget{exploratieve-data-analyse}{%
\subsubsection*{Exploratieve data analyse}\label{exploratieve-data-analyse}}
\addcontentsline{toc}{subsubsection}{Exploratieve data analyse}

\begin{itemize}
\tightlist
\item
  Exploratieve analyse focust op het verkennen van de data en het zoeken naar interessante patronen en afwijkingen van deze patronen.
\item
  Net als bij descriptieve data analyse zal exploratieve analyse de beschikbare data beschrijven en zeggen de resultaten \textbf{niets} over ongeobserveerde feiten.
\item
  In tegenstelling tot bij descriptieve data analyse, gaat exploratieve data analyse verder dan het louter beschrijven van de data en tracht men interessante patronen te ontdekken in de data.
\item
  Voorbeelden:

  \begin{itemize}
  \tightlist
  \item
    Zijn er specifieke kenmerken van studenten die sterk gerelateerd zijn aan hun studieresultaten.
  \item
    Zijn er opmerkelijke verschillen tussen vakken wat betreft de punten die behaald worden. Zo ja, wat zijn dan deze verschillen.
  \item
    Zijn er producten in ons gamma die gevoelig zijn voor seizoenseffecten?
  \end{itemize}
\item
  Je kan een exploratieve data analyse vergelijken met het werk van een detective die als taak heeft verbanden te ontdekken tussen verschillende bewijsstukken om zo inzicht te verschaffen wat er gebeurd is tijdens de misdaad.
\end{itemize}

\hypertarget{confirmatorische-data-analyse}{%
\subsubsection*{Confirmatorische data analyse}\label{confirmatorische-data-analyse}}
\addcontentsline{toc}{subsubsection}{Confirmatorische data analyse}

\begin{itemize}
\tightlist
\item
  Confirmatorische analyse focust op het bevestigen of weerleggen van vermoedens die men heeft met behulp van de beschikbare data.
\item
  In tegenstelling tot descriptieve en exploratieve data analyse zal men bij confirmatorische data analyse wel conclusies trekken die verder gaan dan de geobserveerde data.
\item
  Omdat confirmatorische data analyses ook uitspraken doen over ongeobserveerde data, is er altijd een mate van onzekerheid over de correctheid van de resultaten.
\item
  Voorbeelden:

  \begin{itemize}
  \tightlist
  \item
    Halen studenten met 8u Wiskunde achtergrond betere resultaten dan studenten met 6u Wiskunde achtergrond? In welke mate zijn we zeker dat dit voor alle studenten geldt en niet enkel voor de studenten waarover we data hebben?
  \item
    Verkoopt product X beter bij mannen dan bij vrouwen? In welke mate zijn we zeker dat dit verschil niet een toevalligheid in de data is?
  \end{itemize}
\item
  Je kan een confirmatorische data analyse vergelijken met het werk van een rechter die op basis van het aangeboden bewijsmateriaal moet beslissen of er genoeg bewijs is om iemand te veroordelen van de misdaad.
\end{itemize}

\hypertarget{predictieve-data-analyse}{%
\subsubsection*{Predictieve data analyse}\label{predictieve-data-analyse}}
\addcontentsline{toc}{subsubsection}{Predictieve data analyse}

\begin{itemize}
\tightlist
\item
  Het doel van predictieve analyse is om op basis van de beschikbare data voorspellingen te doen over de toekomst of over nieuwe of alternatieve situaties.
\item
  Net als bij confirmatorische data analyse zal predictieve data analyse uitspraken doen die ook van toepassing zijn voor ongeobserveerde feiten/situaties.
\item
  Bijgevolg is er net als bij confirmatorische data analyse dus een zekere onzekerheid over de conclusies die men trekt.
\item
  Voorbeelden:

  \begin{itemize}
  \tightlist
  \item
    Zal een studente die met meer dan 80\% haar diploma van het middelbaar onderwijs behaalt slagen in eerste zit voor het vak \emph{Exploratieve en Descriptieve Data Analyse}?
  \item
    Zullen de verkoopcijfers van product Y het komende jaar verder stijgen en met hoeveel procent?
  \end{itemize}
\item
  Je kan een predictieve data analyse vergelijken met het werk van een detective die op basis van het bewijsmateriaal op een misdaadscene moet voorspellen waar en wanneer de dader opnieuw zal toeslaan.
\end{itemize}

\hypertarget{rol-van-data-in-de-bedrijfswereld}{%
\subsection{Rol van data in de bedrijfswereld}\label{rol-van-data-in-de-bedrijfswereld}}

\begin{itemize}
\tightlist
\item
  Er zijn verschillende redenen waarom bedrijven data bijhouden. Deze kunnen we onderverdelen in volgende categorieën: Geschiedenis bijhouden, beslissingen nemen en voorspellingen maken.
\end{itemize}

\hypertarget{geschiedenis-bijhouden}{%
\subsubsection*{Geschiedenis bijhouden}\label{geschiedenis-bijhouden}}
\addcontentsline{toc}{subsubsection}{Geschiedenis bijhouden}

\begin{itemize}
\tightlist
\item
  Je registreert feiten zodat je achteraf met zekerheid kunt weten wat de realiteit in het verleden was.
\item
  Dit is belangrijk als je wilt evalueren of een bedrijf goed beheerd wordt. Hiervoor heb je inzicht in het verleden nodig.
\item
  De gegevens die worden bijgehouden in een boekhouding en jaarrekeningen zijn hier een typisch voorbeeld van.
\end{itemize}

\hypertarget{dagelijkse-werking}{%
\subsubsection*{Dagelijkse werking}\label{dagelijkse-werking}}
\addcontentsline{toc}{subsubsection}{Dagelijkse werking}

\begin{itemize}
\tightlist
\item
  Opdat een bedrijf zijn dagelijkse werking kan uitvoeren, is het essentieel een up to date zicht te hebben van de werkelijkheid. Als een klant belt met een klacht over een levering, dan moet je als onderneming kunnen achterhalen wat de klant precies besteld heeft, of dit reeds geleverd is, of de klant al betaald heeft, enzovoort. Zonder deze informatie kan een onderneming haar dagelijkse werking niet garanderen.
\item
  Om de dagelijkse werking te verzekeren, hebben bedrijven altijd al data bijgehouden. Denk maar aan informatie over aankoop- en verkooporders, de financiële gegevens in de boekhouding, de afschriften van een bank, de productieplanning, enzovoort.
\end{itemize}

\hypertarget{beslissingen-nemen}{%
\subsubsection*{Beslissingen nemen}\label{beslissingen-nemen}}
\addcontentsline{toc}{subsubsection}{Beslissingen nemen}

\begin{itemize}
\tightlist
\item
  Een bedrijf neemt dagelijks talrijke beslissingen op verschillende niveaus

  \begin{itemize}
  \tightlist
  \item
    Operationeel.

    \begin{itemize}
    \tightlist
    \item
      Vb: Moet ik een nieuwe bestelling plaatsen voor grondstof X of hebben we nog genoeg voorraad?
    \item
      Dit zijn typisch zeer frequente beslissingen die nodig zijn om de dagelijkse werking te garanderen.
    \item
      Deze beslissingen worden genomen door mensen op de werkvloer of door het (lager) management.
    \end{itemize}
  \item
    Tactisch/Management.

    \begin{itemize}
    \tightlist
    \item
      Vb: Sluit ik best een exclusief contract af met 1 leverancier voor grondstof X voor een vaste periode en tegen een vaste verkoopsprijs of koop ik wanneer nodig tegen de marktprijs?
    \item
      Deze beslissingen worden minder frequent genomen dan operationele beslissingen en zijn typisch nodig om de werking van de onderneming op middellange termijn te optimaliseren.
    \item
      Deze beslissingen worden genomen door het management van een onderneming en hebben een aanzienlijke impact.
    \end{itemize}
  \item
    Strategisch.

    \begin{itemize}
    \tightlist
    \item
      Vb: Zullen we grondstof X aankopen op de markt of beslissen we deze grondstof zelf te produceren?
    \item
      Deze beslissingen hebben een zeer grote impact op de onderneming en worden niet frequent genomen. Ze vergen typisch ook lange voorbereidingstijd en bepalen de richting en toekomst van de onderneming op lange termijn.
    \item
      Deze beslissingen worden genomen door het topmanagement van een onderneming.
    \end{itemize}
  \end{itemize}
\item
  Data kan bedrijven helpen bij het nemen van beslissingen.

  \begin{itemize}
  \tightlist
  \item
    Dit betekent echter niet dat beslissingen enkel en alleen op data gebaseerd zijn.
  \item
    Vaak wordt data gecombineerd met ervaring en expertise om een beslissing te nemen.
  \end{itemize}
\item
  Bij het nemen van beslissingen op basis van data, kunnen we zowel patronen in historische data gebruiken alsook voorspellingen op basis van data.
\end{itemize}

\hypertarget{producten}{%
\subsubsection*{Producten}\label{producten}}
\addcontentsline{toc}{subsubsection}{Producten}

\begin{itemize}
\tightlist
\item
  Data als inherent onderdeel van een product

  \begin{itemize}
  \tightlist
  \item
    Social media
  \item
    Netflix
  \item
    Spotify
  \item
    Google
  \item
    Uber
  \item
    Deliveroo
  \item
    \ldots{}
  \end{itemize}
\item
  Data heeft niet langer puur ondersteunende rol
\end{itemize}

\hypertarget{voorbeeld-netflix}{%
\paragraph{Voorbeeld: Netflix}\label{voorbeeld-netflix}}

\begin{itemize}
\tightlist
\item
  Netflix Prize (2006)

  \begin{itemize}
  \tightlist
  \item
    Wereldwijde open competitie voor de constructie van een nieuw algoritme dat moest voorspellen hoe goed een klant een film zou beoordelen op basis van zijn of haar filmvoorkeuren.
  \item
    Winnaar was het team dat als eerste een verbetering van 10\% kon realiseren ten opzichte van het algoritme van Netflix zelf.
  \item
    Eerste prijs was 1 miljoen USD.
  \item
    Hiervoor stelde Netflix een dataset ter beschikking met 100 miljoen filmbeoordelingen van 500 000 klanten met betrekking tot 18 000 films.
  \end{itemize}
\item
  Het kunnen voorspellen hoe hun klanten gaan reageren op specifieke films/series laat Netflix toe hun aanbod aan films en series te optimaliseren om het huidige klantenbestand te behouden en nieuwe klanten aan te trekken.
\item
  De hoeveelheid data die door Netflix wordt verzameld is enorm.

  \begin{itemize}
  \tightlist
  \item
    In 2016 had Netflix 93.8 miljoen leden.
  \item
    Netflix weet wanneer je pauzeert.
  \item
    Netflix weet op welke dagen en welke uren je kijkt.
  \item
    Netflix weet wat je kijkt.
  \item
    Netflix weet van waar je kijkt.
  \item
    Netflix weet op welk soort toestellen je kijkt.
  \item
    Netflix weet wanneer je definitief stopt met het bekijken van een serie.
  \item
    Netflix weet hoe snel je verschillende afleveringen van een serie achter elkaar kijkt.
  \item
    Netflix weet welke titels je zoekt.
  \end{itemize}
\item
  Netflix komt op deze manier zeer veel te weten over het kijkgedrag van zijn klanten en kan op basis van deze inzichten betere beslissingen nemen. Bijvoorbeeld:

  \begin{itemize}
  \tightlist
  \item
    Netflix ontdekt uit haar data dat 40\% van haar klanten een serie zijn beginnen te kijken die door het oorspronkelijke productiehuis is stopgezet.
  \item
    Stel dat Netflix uit de data ook ontdekt dat 85\% van deze klanten de serie volledig uitkijken zonder dat het tempo waartegen men afleveringen kijkt significant afneemt.
  \item
    Op basis van deze inzichten kan Netflix eventueel beslissen om de rechten van de serie te kopen (die goedkoop zullen zijn aangezien de serie was stopgezet) en zelf een nieuw seizoen voor de serie te maken.
  \end{itemize}
\item
  House of Cards

  \begin{itemize}
  \tightlist
  \item
    Netflix deed het beste bod voor de serie House of Cards waardoor het won van kanalen zoals HBO.
  \item
    Ze kochten initieel 2 seizoenen van de serie waar een prijskaartje aan vast hing van meer dan 100 miljoen dollar.
  \item
    Deze beslissing was voor een groot stuk gebaseerd op data:

    \begin{itemize}
    \tightlist
    \item
      Netflix leerde uit haar data dat haar klanten geïnteresseerd waren in producties van regiseur David Fincher.
    \item
      Netflix leerde uit haar data dat haar klanten geïnteresseerd waren in de oorspronkelijke Britse versie van House of Cards.
    \item
      Netflix leerde uit haar data dat haar klanten geïnteresseerd waren in producties met Kevin Spacey.
    \end{itemize}
  \item
    Maar ook na de beslissing om deze serie te maken, bleef Netflix haar data gebruiken om slimme beslissingen te nemen.

    \begin{itemize}
    \tightlist
    \item
      Er werden verschillende trailers gemaakt en afhankelijk van je voorkeuren kreeg je een trailer op maat te zien.
    \item
      Klanten die vooral graag Kevin Spacey zagen, kregen een trailer waar vooral Kevin Spacey in voorkwam.
    \item
      Klanten die vooral geïnteresseerd waren in films van David Fincher, kregen een trailer te zien die de typische ``look\&feel'' had van David Fincher.
    \item
      Klanten die ook de Britse versie hadden gezien, kregen een trailer te zien de vooral op het verhaal focuste.
    \end{itemize}
  \end{itemize}
\end{itemize}

\hypertarget{data-revoluties}{%
\subsection{Data revoluties}\label{data-revoluties}}

\begin{itemize}
\tightlist
\item
  Data over de maatschappij

  \begin{itemize}
  \tightlist
  \item
    Het verzamelen van data is iets dat teruggaat tot in de oudheid.
  \item
    Denk hierbij aan de volkstellingen die reeds plaatsvonden ten tijden van de Romeinen.
  \item
    Een volkstelling gaat alle inwoners van een bevolking registreren, samen met diverse kenmerken zoals burgelijke status, leeftijd, geslacht, enzovoort.
  \item
    Volkstellingen waren en zijn nog steeds belangrijk voor een overheid om de impact van haar openbaar beleid te kunnen inschatten.
  \end{itemize}
\item
  Scientific Management

  \begin{itemize}
  \tightlist
  \item
    Frederick Taylor
  \item
    Eind 19de eeuw
  \item
    Benaderde het organiseren van werk op een wetenschappelijke manier.
  \item
    Ging data verzamelen om vervolgens te analyseren hoe men werk efficiënter kon organiseren.
  \item
    Een van de eerste vormen van dataverzameling en -analyse om bedrijfswaarde (productiviteit) te creëren.
  \item
    Beperkt in hoeveelheid data omdat registratie en analyse nog manueel gebeurde.
  \end{itemize}
\item
  Het ontstaan van het digitale tijdperk

  \begin{itemize}
  \tightlist
  \item
    Met de uitvinding van de computer tijdens en na de tweede wereldoorlog, is de mensheid het digitale tijdperk ingegaan.
  \item
    De computer zorgt ervoor dat we data in een digitale vorm (als een reeks van één en nullen) opslaan. Dit biedt het voordeel dat exacte kopieën van de data gemaakt kunnen worden met één muisklik.
  \end{itemize}
\item
  Digitalisatie van de werkvloer

  \begin{itemize}
  \tightlist
  \item
    Computers op de werkvloer dateert terug tot midden vorige eeuw, maar de grote doorbraak komt er met de opkomst van de personal computer

    \begin{itemize}
    \tightlist
    \item
      1977: Apple Home Computer II
    \item
      1981: IBM Personal Computer
    \item
      Eind jaren 80, begin jaren 90 was de PC wijdverspreid op de werkvloer.
    \item
      Dit liet toe meer data te registreren, maar deze was nog moeilijk te delen met andere computers.
    \end{itemize}
  \item
    Opkomst Internet/WWW in de bedrijfswereld

    \begin{itemize}
    \tightlist
    \item
      1990: De technologie voor WWW werd publiek gedeeld door Tim Berners-Lee.
    \item
      Dankzij WWW en internettechnologie werd het steeds eenvoudiger om digitaal werk te delen.
    \end{itemize}
  \item
    Opkomst van e-commerce

    \begin{itemize}
    \tightlist
    \item
      1995: Begin van dot-com bubble/hype.
    \item
      Opkomst van digitale ondernemingen (vb. Amazon, Netflix, Google, \ldots).
    \item
      Digitale handel maakt het eenvoudiger om gegevens hierover te registreren.
    \end{itemize}
  \end{itemize}
\item
  Digitalisatie van mensen

  \begin{itemize}
  \tightlist
  \item
    Opkomst Web 2.0 (begin 2000)

    \begin{itemize}
    \tightlist
    \item
      Inhoud van het web wordt nu gecreëerd door de bezoekers/gebruikers/klanten.
    \item
      Websites worden dynamisch (passen zich aan de context en bezoeker aan).
    \end{itemize}
  \item
    Opkomst sociale media

    \begin{itemize}
    \tightlist
    \item
      Gebruikers gaan spontaan hun leven digitaliseren.
    \item
      Hiervoor worden diverse media gebruikt (foto, video, tekst, \ldots).
    \item
      Facebook, Twitter, Instagram, Persoonlijke blogs, \ldots{} .
    \item
      Nog nooit heeft zo'n groot deel van de wereldbevolking informatie gecreëerd en gedeeld met de rest van de wereld.
    \end{itemize}
  \end{itemize}
\item
  Digitalisatie van dingen

  \begin{itemize}
  \tightlist
  \item
    Opkomst goedkope sensoren
  \item
    Steeds meer ``dingen'' (machines, auto's, huishoudtoestellen, huizen, steden, \ldots) worden `inteligent'.
  \item
    Internet of Things (IoT): Al deze intelligente dingen worden via het Internet met elkaar verbonden.
  \item
    De hoeveelheid data die hiermee gegenereerd zal worden is ongezien.
  \item
    Volgens IDC studie waren in 2013 reeds 7\% van de ``verbindbare dingen'' geconnecteerd aan het Internet of Things.
  \item
    In dezelfde studie voorspellen ze dat dit zal stijgen tot 15\% in 2020.
  \item
    In 2013 werd 2\% van alle data in het digitaal universum geproduceerd door het IoT.
  \item
    Verwacht wordt dat dit zal stijgen tot 10\% in 2020.
  \end{itemize}
\end{itemize}

\hypertarget{data-explosion}{%
\subsection{Data explosion}\label{data-explosion}}

\begin{itemize}
\tightlist
\item
  De hoeveelheid data die de laatste decennia gegenereerd en opgeslagen wordt is enorm toegenomen.
\item
  Deze groei is exponentieel (de groei gaat steeds sneller). Meer specifiek verdubbelt de hoeveelheid data in het digitaal universum iedere 2 jaar.
\item
  In 2018 bestond het digitaal universum uit 33 Zetabytes data

  \begin{itemize}
  \tightlist
  \item
    1 Zetabyte = 1024 Exabytes
  \item
    1 Exabyte = 1024 Petabytes
  \item
    1 Petabyte = 1024 Terabytes
  \item
    1 Terabyte = 1024 Gigabytes
  \end{itemize}
\item
  Volgens studies zal het digitaal universum in 2035 uit 2142 Zetabytes bestaan.

  \begin{itemize}
  \tightlist
  \item
    Dit is een toename van ca. 28\% per jaar, i.e.~een verdubbeling elke 3 jaar.
  \end{itemize}
\end{itemize}

\hypertarget{waarover-verzamelen-bedrijven-data}{%
\subsection{Waarover verzamelen bedrijven data}\label{waarover-verzamelen-bedrijven-data}}

\begin{itemize}
\tightlist
\item
  Het ultieme doel van een onderneming is gegevens te verzamelen die hen toelaten om het gedrag van hun omgeving beter te begrijpen, alsook de werking van hun eigen onderneming.
\item
  Onder omgeving verstaan we:

  \begin{itemize}
  \tightlist
  \item
    Klanten
  \item
    Concurrenten
  \item
    Leveranciers
  \item
    Alternatieve markten
  \item
    Overheden
  \end{itemize}
\item
  Onder werking van eigen onderneming vertaan we o.a.:

  \begin{itemize}
  \tightlist
  \item
    Werknemers
  \item
    Processen
  \item
    Producten
  \item
    Diensten
  \end{itemize}
\end{itemize}

\hypertarget{van-data-tot-actionable-insights}{%
\subsection{Van data tot `actionable insights'}\label{van-data-tot-actionable-insights}}

\begin{itemize}
\tightlist
\item
  Data

  \begin{itemize}
  \tightlist
  \item
    Data verwijst typisch naar de gegevens die geregistreerd en opgeslagen worden.
  \item
    Data beschrijft een heel klein aspect van een realiteit (bijvoorbeeld op welk exact tijdstip ben ik aflevering 2 van ``House of Cards'' beginnen te kijken).
  \item
    Data op zich heeft echter heel weinig waarde.
  \end{itemize}
\item
  Informatie

  \begin{itemize}
  \tightlist
  \item
    Als we echter data gaan analyseren, dan kunnen we dit transformeren tot informatie.
  \item
    Informatie beschrijft een realiteit en gaat typisch op zoek naar patronen in de data en afwijkingen op deze patronen.
  \item
    Bijvoorbeeld: Ik kijk typisch House of Cards gedurende de week om 20u00 's avonds, maar stop meestal met kijken om 20u30, waardoor ik in de week zelden een aflevering in 1 keer uitkijk.
  \item
    Informatie is beschrijvend en zegt ons WAT de realiteit is.
  \end{itemize}
\item
  Actionable Insights

  \begin{itemize}
  \tightlist
  \item
    Actionable Insights is informatie die ons niet enkel zegt WAT de realiteit is, maar ons ook het inzicht verschaft HOE we moeten handelen.
  \item
    Niet alle informatie is actionable.
  \item
    Op basis van actionable insights en in combinatie met onze eigen ervaringen en kennis die we reeds bezitten, komen we soms tot inzichten die beschrijven HOE we moeten handelen.
  \end{itemize}
\end{itemize}

\hypertarget{data-scientists}{%
\subsection{Data Scientists}\label{data-scientists}}

\begin{itemize}
\tightlist
\item
  Nieuwe jobomschrijving.
\item
  Verantwoordelijk om data te transformeren naar `actionable insights' en hier iets mee te doen om bedrijfswaarde te creëren.
\item
  Omschreven als meest `sexy job' van de 21ste eeuw door HBR

  \begin{itemize}
  \tightlist
  \item
    Opvolgers van de Wall Street `Quants' uit de jaren 80 en 90.
  \end{itemize}
\item
  Vaardigheden

  \begin{itemize}
  \tightlist
  \item
    Bedrijfskunde

    \begin{itemize}
    \tightlist
    \item
      Productontwikkeling
    \item
      Management
    \end{itemize}
  \item
    Machine Learning / Big Data

    \begin{itemize}
    \tightlist
    \item
      Ongestructureerde data
    \item
      Gestructureerde data
    \item
      Machine Learning
    \item
      Big Data
    \end{itemize}
  \item
    Wiskunde en Operationeel Onderzoek

    \begin{itemize}
    \tightlist
    \item
      Optimalisatie
    \item
      Wiskunde
    \item
      Simulatie
    \end{itemize}
  \item
    Programmeren
  \item
    Statistiek

    \begin{itemize}
    \tightlist
    \item
      Visualisatie
    \item
      Tijdreeksanalyse
    \item
      Wetenschappelijk onderzoek
    \item
      Data Manipulatie
    \end{itemize}
  \end{itemize}
\item
  4 profielen van data scientists

  \begin{itemize}
  \tightlist
  \item
    Data Businessperson

    \begin{itemize}
    \tightlist
    \item
      Focust voornamelijk hoe data omzet kan genereren.
    \item
      Vaak in een leidinggevende rol.
    \item
      Werken zelf ook met data en beschikken over de nodige technische vaardigheden.
    \end{itemize}
  \item
    Data Creatives

    \begin{itemize}
    \tightlist
    \item
      Zijn in staat een volledige data analyse zelfstandig uit te voeren.
    \item
      Hebben een hele brede bagage aan technische vaardigheden.
    \item
      Beschikken in zekere mate over bedrijfskundige vaardigheden.
    \item
      Gaan vaak innovatief om met data.
    \end{itemize}
  \item
    Data Developer

    \begin{itemize}
    \tightlist
    \item
      Is voornamelijk gefocust op de technische uitdagingen met betrekking tot het beheer van data.
    \item
      Sterke programmeervaardigheden. Zijn in staat productie-code te schrijven.
    \item
      Zijn sterk in het gebruik van machine learning technieken.
    \end{itemize}
  \item
    Data Researcher

    \begin{itemize}
    \tightlist
    \item
      Vaak mensen met een wetenschappelijke achtergrond (doctoraat).
    \item
      Sterk in statistische vaardigheden en wetenschappelijk onderzoek.
    \end{itemize}
  \end{itemize}
\end{itemize}

\hypertarget{de-kunst-van-data-analyse}{%
\subsection{De kunst van data analyse}\label{de-kunst-van-data-analyse}}

\begin{itemize}
\tightlist
\item
  Data analyse is een kunst. Net als bij iedere kunst, kunnen we hierbij drie componenten onderscheiden: kennis en vaardigheden, ervaring en creativiteit.
\item
  Kennis en vaardigheden

  \begin{itemize}
  \tightlist
  \item
    Als data analist moet je de juiste hulpmiddelen kunnen identificeren voor het voorgelegde probleem.
  \item
    Deze diverse hulpmiddelen moet je zo goed mogelijk beheersen.
  \item
    Bij (exploratieve) data analyse gaat het hierbij zowel over analysetechnieken als over datavaardigheden.
  \item
    Dit aspect kun je leren en laat je reeds toe om correcte analyses uit te voeren.
  \end{itemize}
\item
  Ervaring

  \begin{itemize}
  \tightlist
  \item
    Hoe meer data je analyseert, hoe beter je er in wordt.
  \item
    Ook laat ervaring toe om sneller vaste patronen in je werk te herkennen en efficiënter te worden in wat je doet.
  \item
    Ervaring is ook essentieel om complexere uitdagingen beheersbaar te maken.
  \item
    Dit deel kunnen we je niet `leren', maar heb je wel volledig in de hand.
  \end{itemize}
\item
  Creativiteit

  \begin{itemize}
  \tightlist
  \item
    Een kunstenaar die over kennis, vaardigheden en ervaring beschikt, maar creativiteit ontbreekt, kan perfecte replica's maken van een kustwerk, mar kan zelf geen nieuwe kunst creëren.
  \item
    Creativiteit is in staat zijn op een nieuwe en onverwachte manier naar data te kijken en deze te visualiseren.
  \item
    Het is niet zeker dat dit aspect aan te leren is. Maar dit hoeft niet te verhinderen dat je een goede data scientist wordt, zolang je maar voldoende aandacht besteedt aan de andere twee componenten.
  \end{itemize}
\end{itemize}

\hypertarget{de-kracht-van-descriptieve-en-exploratieve-data-analyse}{%
\subsection{De kracht van descriptieve en exploratieve data analyse}\label{de-kracht-van-descriptieve-en-exploratieve-data-analyse}}

\url{https://www.youtube.com/watch?v=RUwS1uAdUcI}

\hypertarget{data-data-types}{%
\section{Data \& Data types}\label{data-data-types}}

\begin{itemize}
\tightlist
\item
  Data is het resultaat van een meting van een attribuut van een specifiek object met een specifiek meetinstrument.

  \begin{itemize}
  \tightlist
  \item
    Het object verwijst naar wat je gaat meten.

    \begin{itemize}
    \tightlist
    \item
      vb.: Student ``Karel Jespers''.
    \end{itemize}
  \item
    Een object hoort meestal tot een verzameling van objecten. Deze verzameling wordt ook wel de populatie genoemd.

    \begin{itemize}
    \tightlist
    \item
      vb.: Populatie ``Studenten 1ste Ba HI/BI''.
    \end{itemize}
  \item
    Een specifiek object uit de populatie wordt ook wel element genoemd.

    \begin{itemize}
    \tightlist
    \item
      vb.: ``Karel Jespers'' is een element uit de populatie ``Student 1ste Ba HI/BI''.
    \end{itemize}
  \item
    Je meet altijd een specifiek aspect van het object. Omdat de meetwaarde van dit aspect kan variëren tussen verschillende objecten (elementen) in je verzameling (populatie), worden zulke aspecten ook variabelen genoemd.

    \begin{itemize}
    \tightlist
    \item
      vb.: Lengte is een specifiek aspect (variabele) van de student ``Karel Jespers'' (element).
    \end{itemize}
  \item
    De meting gebeurt met behulp van een meetinstrument. Het is belangrijk te beseffen dat een meetinstrument altijd een zekere nauwkeurigheid heeft (tot hoeveel cijfers na de komma exact kan je meten?) en mogelijk ook onderhevig kan zijn aan willekeurige en/of systematische meetfouten.

    \begin{itemize}
    \tightlist
    \item
      vb.: Student ``Karel Jespers'' wordt gemeten met een meetlat bevestigd tegen de muur. De meetlat heeft een nauwkeurigheid van 1cm, dus we kunnen zijn lengte niet uitdrukken in millimeters. Verder is de meetlat 2cm te laag opgehangen. Bijgevolg is er een systematische meetfout van 2cm. Tenslotte wordt de meting geregistreerd door een arts die vluchtig kijkt waar de student uitkomt op de meetlat. Het is dus niet onmogelijk dat de werkelijke lengte (willekeurig) afwijkt van de geregistreerde lengte.
    \item
      Tenzij anders vermeld wordt, gaan we in dit hoofdstuk uit van meetinstrumenten met oneindige nauwkeurigheid en zonder meetfouten.
    \end{itemize}
  \item
    De uitkomst van een meting voor een specifiek element wordt de waarde genoemd.

    \begin{itemize}
    \tightlist
    \item
      vb.: 1m80 is de waarde van de variabele ``lengte'' voor element ``student Karel Jespers''
    \end{itemize}
  \end{itemize}
\end{itemize}

\hypertarget{dataset}{%
\subsection{Dataset}\label{dataset}}

\begin{itemize}
\tightlist
\item
  Een dataset is een verzameling van data waarbij

  \begin{itemize}
  \tightlist
  \item
    Iedere rij één element uit de populatie voorstelt.
  \item
    Iedere kolom een variabele is die gemeten wordt.
  \item
    De verschillende rijen verschillende elementen uit dezelfde populatie voorstellen.
  \item
    De waarde in een cel de meting is van de betreffende variabele voor het betreffend element.
  \end{itemize}
\end{itemize}

\begin{table*}

\caption{\label{tab:2-2}
Uitgaande vluchten NYC}
\centering
\fontsize{6}{8}\selectfont
\begin{tabu} to \linewidth {>{\raggedright}X>{\raggedright}X>{\raggedright}X>{\raggedleft}X}
\toprule
luchthaven & maatschappij & datum & vertrek\_vertraging\\
\midrule
EWR & United Air Lines Inc. & 2013-01-01 05:15:00 & 2\\
LGA & United Air Lines Inc. & 2013-01-01 05:29:00 & 4\\
JFK & American Airlines Inc. & 2013-01-01 05:40:00 & 2\\
LGA & Delta Air Lines Inc. & 2013-01-01 06:00:00 & -6\\
EWR & United Air Lines Inc. & 2013-01-01 05:58:00 & -4\\
\addlinespace
EWR & JetBlue Airways & 2013-01-01 06:00:00 & -5\\
LGA & ExpressJet Airlines Inc. & 2013-01-01 06:00:00 & -3\\
JFK & JetBlue Airways & 2013-01-01 06:00:00 & -3\\
LGA & American Airlines Inc. & 2013-01-01 06:00:00 & -2\\
JFK & JetBlue Airways & 2013-01-01 06:00:00 & -2\\
\bottomrule
\end{tabu}
\end{table*}

\hypertarget{klassieke-datatypologie}{%
\subsection{Klassieke datatypologie}\label{klassieke-datatypologie}}

\begin{itemize}
\tightlist
\item
  Klassieke onderverdeling van data

  \begin{itemize}
  \tightlist
  \item
    Nominaal, Ordinaal, Interval en Ratio
  \item
    Gebaseerd op de publicatie ``On the Theory of Scales of Measurement'' (1946)

    \begin{itemize}
    \tightlist
    \item
      Beschrijft een hiërarchie van `datatypes'

      \begin{itemize}
      \tightlist
      \item
        Alles wat ordinaal is, is ook nominaal, maar niet omgekeerd.
      \item
        Alles wat interval is, is ook ordinaal, maar niet omgekeerd.
      \item
        Alles wat ratio is, is ook interval, maar niet omgekeerd.
      \end{itemize}
    \item
      Identificeert geschikte statistische testen voor ieder type.
    \end{itemize}
  \end{itemize}
\item
  Ieder datatype voldoet aan één of meerdere van de volgende eigenschappen:

  \begin{itemize}
  \tightlist
  \item
    Identiteit: Iedere waarde heeft een unieke betekenis.
  \item
    Grootorde: Er is een natuurlijke volgorde tussen de waarden.
  \item
    Gelijke intervals: Eenheidsverschillen zijn overal even groot. Dus het verschil tussen 1 en 2 is even groot als het verschil tussen 19 en 20.
  \item
    Absoluut nulpunt: De waarde 0 betekent dat er ook feitelijk niets aanwezig is van de variabele en is niet een arbitrair gekozen nulpunt.
  \end{itemize}
\end{itemize}

\hypertarget{nominaal}{%
\subsubsection*{Nominaal}\label{nominaal}}
\addcontentsline{toc}{subsubsection}{Nominaal}

\begin{itemize}
\tightlist
\item
  Voorbeelden:

  \begin{itemize}
  \tightlist
  \item
    Geslacht: Man, Vrouw.
  \item
    Ondernemingsvorm: vzw, bvba, nv.
  \end{itemize}
\item
  Voldoet enkel aan de eigenschap `identiteit'.
\item
  Dit betekent dat we enkel concluderen of twee waardes gelijk zijn of niet. Er bestaat geen natuurlijke volgorde tussen de verschillende waardes.
\end{itemize}

\hypertarget{ordinaal}{%
\subsubsection*{Ordinaal}\label{ordinaal}}
\addcontentsline{toc}{subsubsection}{Ordinaal}

\begin{itemize}
\tightlist
\item
  Voorbeeld:

  \begin{itemize}
  \tightlist
  \item
    Opleidingsniveau: Lager onderwijs, Middelbaar onderwijs, Hoger onderwijs.
  \item
    Klantentevredenheid: Ontevreden, Matig tevreden, Tevreden, Zeer tevreden.
  \end{itemize}
\item
  Voldoet aan de eigenschappen `identiteit' en `grootorde'.
\item
  Dit betekent dat we niet alleen kunnen concluderen of twee waardes gelijk zijn of niet. Het is ook mogelijk te bepalen welke waarde `groter' is.
\item
  We kunnen echter niet zeggen hoeveel groter één waarde is dan de andere.
\end{itemize}

\hypertarget{interval}{%
\subsubsection*{Interval}\label{interval}}
\addcontentsline{toc}{subsubsection}{Interval}

\begin{itemize}
\tightlist
\item
  Voorbeeld:

  \begin{itemize}
  \tightlist
  \item
    Temperatuur (Celsius).
  \end{itemize}
\item
  Voldoet aan de eigenschappen `identiteit', `grootorde' en `gelijke intervals'.
\item
  We kunnen nu twee waardes vergelijken, bepalen welke groter is alsook de verschillen tussen waardes met elkaar vergelijken.

  \begin{itemize}
  \tightlist
  \item
    We kunnen dus stellen dat het verschil tussen 8 en 9 graden Celsius daadwerkelijk minder groot is dan het verschil tussen 12 en 20 graden Celsius.
  \end{itemize}
\end{itemize}

\hypertarget{ratio}{%
\subsubsection*{Ratio}\label{ratio}}
\addcontentsline{toc}{subsubsection}{Ratio}

\begin{itemize}
\tightlist
\item
  Voorbeeld:

  \begin{itemize}
  \tightlist
  \item
    Gewicht
  \end{itemize}
\item
  Voldoet aan alle 4 de eigenschappen.
\item
  We kunnen verschillende gewichten met elkaar vergelijken, we kunnen bepalen wat zwaarder is en we kunnen gewichtsverschillen onderling vergelijken. Hierbij komt nu ook nog dat we kunnen zeggen hoeveel keer iets zwaarder is dan iets anders.
\item
  Dit is een gevolg van het feit dat de waarde 0 nu feitelijk betekent dat iets geen gewicht heeft.
\end{itemize}

\hypertarget{de-klassieke-datatypologie-is-misleidend}{%
\subsection{De klassieke datatypologie is misleidend}\label{de-klassieke-datatypologie-is-misleidend}}

\begin{itemize}
\tightlist
\item
  Voorbeeld:

  \begin{itemize}
  \tightlist
  \item
    Op een feestje wordt bij het binnengaan oplopende nummers toegewezen aan iedere gast, beginnend bij 1.
  \item
    Tijdens het feestje wordt er een tombola georganiseerd en wie nummer 126 heeft, heeft gewonnen.
  \item
    1 gast vergelijkt dit nummer met haar kaartje en ziet dat ze gewonnen heeft. Zij beschouwde de waarde op haar ticket dus als een nominale variabele want het enige wat ze vergelijkt is of de waarde op haar ticket verschillend is van de winnende waarde.
  \item
    Een andere gast kijkt naar zijn kaartje en ziet dat hij nummer 56 heeft. Hij concludeert dat hij te vroeg is binnengekomen en beschouwt de waarde op zijn kaartje dus als ordinaal.
  \item
    Nog een andere gast heeft een kaartje met nummer 70 en beschikt over bijkomende data omtrent het ritme waarmee gasten zijn binnengekomen. Deze gast kan dus schatten hoeveel later hij had moeten binnenkomen om te winnen en interpreteert zijn nummer dus als een interval variabele.
  \end{itemize}
\item
  Dit voorbeeld illustreert dat het datatype niet een vaststaand kenmerk is van de data, maar afhankelijk is van de vraag die je tracht te beantwoorden en de extra informatie waarover je beschikt.
\end{itemize}

\hypertarget{alternatieve-datatypologie}{%
\subsection{Alternatieve datatypologie}\label{alternatieve-datatypologie}}

\begin{itemize}
\tightlist
\item
  Alternatieve taxonomie van data

  \begin{itemize}
  \tightlist
  \item
    Graden: vb. academische graad: ``op voldoende wijze'', ``onderscheiding'', ``grote onderscheiding'', \ldots{} (geordende labels)
  \item
    Rangordes: vb. plaats in voetbalklassement: 1, 2, 3, \ldots, 16 (gehele getallen die beginnen bij 1)
  \item
    Fracties: vb. percentage opgenomen verlof: van 0\% tot 100\% (ligt tussen 0 en 1, als percentage uit te drukken).
  \item
    Aantallen: vb aantal kinderen: 0, 1, 2, \ldots{} (niet-negatieve gehele waarden).
  \item
    Hoeveelheden: vb. inkomen (niet-negatieve reële waarden).
  \item
    Saldo: vb. winst (negatieve en positieve reële waarden).
  \end{itemize}
\item
  Voor deze cursus volstaat het meestal een onderscheid te maken tussen nominale en ordinale variabelen (samen ``categorische'' variabelen) en continue variabelen

  \begin{itemize}
  \tightlist
  \item
    Categorisch:

    \begin{itemize}
    \tightlist
    \item
      Nominaal
    \item
      Ordinaal.
    \end{itemize}
  \item
    Continu: (Interval + Ratio)
  \end{itemize}
\end{itemize}

\hypertarget{referenties}{%
\section{Referenties}\label{referenties}}

\begin{enumerate}
\def\labelenumi{\arabic{enumi}.}
\tightlist
\item
  \href{https://hbr.org/2012/10/data-scientist-the-sexiest-job-of-the-21st-century}{Data Scientist, the Sexiest Job of the 21st Centure}
\item
  \href{https://en.wikipedia.org/wiki/Netflix_Prize}{Netflix Prize}
\item
  \href{https://blog.kissmetrics.com/how-netflix-uses-analytics/}{How Netflix Uses Analytics}
\item
  \href{https://www.emc.com/leadership/digital-universe/2014iview/index.htm}{The Digital Universe of Opportunities - website}
\item
  \href{http://bcove.me/9s38pkjm}{The Digital Universe of Opportunities - videoclip}
\item
  \href{http://www.bbc.com/news/magazine-23509153}{How the Computer Changed the Office Forever}
\item
  \href{http://www.ehow.com/about_6362639_history-computers-workplace.html}{History of Computers in the Workplace}
\item
  \href{https://www.geeksforgeeks.org/web-1-0-web-2-0-and-web-3-0-with-their-difference/}{Web 1.0, 2.0, 3.0}
\item
  \href{https://en.wikipedia.org/wiki/File:DIKW_(1).png}{From Data to Understanding}
\item
  \href{http://www.oreilly.com/data/free/files/analyzing-the-analyzers.pdf}{Analyzing the Analyzers}
\item
  \href{http://www.mnestudies.com/research/scales-measurement}{Scales of Measurement}
\item
  \href{http://websites.uwlax.edu/tbrooks/eco307/handouts/velleman\%201993\%20-\%20typologies\%20misleading.pdf}{Nominal, Ordinal, Interval, and Ratio Typologies are Misleading}
\end{enumerate}

\bibliography{book.bib,packages.bib}



\end{document}
